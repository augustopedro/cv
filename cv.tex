% -- Encoding UTF-8 without BOM
% -- XeLaTeX => PDF (BIBER)

\documentclass[a4paper]{cv-style}
\setdefaultlanguage{brazil}
\usepackage{fontawesome}
\usepackage{marvosym}
\hypersetup{
  colorlinks=true,
  linkcolor=blue
}

\begin{document}

\header{Pedro Augusto }{Duarte de Almeida}        % Your name
%\lastupdated

%----------------------------------------------------------------------------------------
%	SIDEBAR SECTION  -- In the aside, each new line forces a line break
%----------------------------------------------------------------------------------------
\begin{aside}
~
%{\includegraphics[scale=0.27]{img/me2.jpg}}
%
\section{Web}
{\small\faLinkedin\space}\underline{\href{https://linkedin.com/in/peadalmeida}{linkedin.com/in/peadalmeida}}
{\small\faGithub\space}\underline{\href{https://github.com/augustopedro}{github.com/augustopedro}}
{\small\faTwitter\space}\underline{\href{https://twitter.com/pedroaugusto}{twitter.com/pedroaugusto}}
~
\section{Idiomas}
\asidelist{\textbf{Português:}}%
{\includegraphics[scale=0.30]{img/star.png}\includegraphics[scale=0.30]{img/star.png}\includegraphics[scale=0.30]{img/star.png}\includegraphics[scale=0.30]{img/star.png}}
\asidelist{\textbf{Inglês:}}%
{\includegraphics[scale=0.30]{img/star.png}\includegraphics[scale=0.30]{img/star.png}\includegraphics[scale=0.30]{img/star.png}\includegraphics[scale=0.30]{img/star_empty.png}}
~
\section{Linguagens}
  C, C\#, Python
  Shell Script
  HTML/CSS/JavaScript
  \LaTeX
~
\section{Frameworks}
  Microsoft .NET
  Django
~
\section{Metodologias}
  Scrum, Kanban
~
\section{Banco de Dados}
  MySQL, IBM DB2
  Microsoft SQL Server
~
\section{Ferramentas}
Visual Studio Code
Sublime Text
GitHub
Microsoft Office
~
\section{Sistemas Operacionais}
  Linux
  Microsoft Windows
~
\section{Interesses}
Análise de Sistemas
Engenharia de Software
Gestão de Projetos
Metodologias Ágeis
Empreendedorismo
Ecossistemas de Startups
~
\end{aside}
%----------------------------------------------------------------------------------------
%	WORK EXPERIENCE SECTION
%----------------------------------------------------------------------------------------
\section{Contato}
  \centerline{{\Large\WritingHand\space}{Rua Nelson Soares de Faria 57/101, Cidade Nova, Belo Horizonte/MG}}
  {\vspace{2.9pt}} 
  \centerline{{\large\faPhone\space}{+55 37 9 8821-4181}\space\faEnvelopeO\space\underline{peadalmeida@gmail.com}}
  {\vspace{2.9pt}}
 
  \begin{center}
    \noindent\rule{13cm}{3.0pt}
  \end{center}

  \section{Experiência}

    \begin{entrylist}
      \vspace{5pt}
      \entry
      {2015 - Atual}
      {Estagiário em Análise de Sistemas}
      {Banco Mercantil do Brasil}
      {Trabalha no setor de Recuperação de Crédito, dando manutenção em código legado e implementando soluções na plataforma .NET com C\#. Paralelamente desenvolve competências como levantamento de requisitos, trabalho em equipe, e gerenciamento de projetos com a prática do Scrum.}
      \vspace{5pt}
      \entry
      {2013 - 2014}
      {Estudante Pesquisador}
      {Universidade Federal de Lavras}
      {Atuou como pesquisador voluntário no projeto de iniciação científica sobre métodos, técnicas e ferramentas para desenvolvimento de software. Durante o projeto teve acesso a ferramentas de medição de software e foi inserido no meio científico, ampliando a noção sobre a ciência e áreas de pesquisa relacionadas à Computação.}
    \end{entrylist}

%----------------------------------------------------------------------------------------
%	EDUCATION SECTION
%----------------------------------------------------------------------------------------
\section{Educação}
  \begin{entrylist}
\entry
{2014 - Atual}
{{\normalfont Bacharelado em} Engenharia de Software}
{PUC Minas}
{\vspace{-0.3cm}}
\entry
{2011 - 2014}
{{\normalfont Bacharelado em} Ciência da Computação}
{UFLA}

%------------------------------------------------
\end{entrylist}

\section{Certificados}
\begin{entrylist}
\entry
{2016}
{CPRE Foundation Certification}
{IREB}
{Certificado Profissional em Engenharia de Requisitos - Nível Foundation}
\entry
{2016}
{TOEFL ITP Assessment Series}
{ETS}
{Certificado de Proficiência em Inglês - Nível B2}
% \entry
% {2017}
% {CTFL Foundation Certification}
% {International Software Testing Qualifications Board}
% {Certificado Profissional em Testes de Software - Nível Foundation}
%------------------------------------------------
\end{entrylist}

%----------------------------------------------------------------------------------------
%	AWARDS SECTION
%----------------------------------------------------------------------------------------

\section{Extracurricular}

  \begin{entrylist}
    \entry
    {2017 - Atual}
    {Estudo do Ecossistema de Startups de Belo Horizonte}
    {PUC Minas}
    {Atualmente desenvolve uma pesquisa de caráter exploratório sobre o ecossistema de startups de tecnologia de Belo Horizonte, com o objetivo específico de conhecê-lo melhor para que seja feita uma avaliação do seu momento atual e maturidade.}
    \entry
    {2015}
    {Clicando na Terceira Idade}
    {PUC Minas}
    {Voluntário no curso de capacitação em informática da PUC Minas focado na atenção aos idosos. Deu apoio nas aulas de informática e na elaboração do material didático.}
    \entry
    {2012}
    {Robocode: Aprendendo a programar de forma lúdica}
    {UFLA}
    {Voluntário no projeto de extensão dando aulas para pessoas da comunidade de Lavras interessadas em aprender programação de computadores, entre elas alunos do ensino médio, graduandos de Ciência da Computação e áreas afins.}
    
  \end{entrylist}
\section{Diversos}
\tikzset{
    cercle/.pic={
      \node [draw, thick, circle, minimum width=10pt] {\tikzpictext};
    },
  }
\hspace*{1mm}
\begin{minipage}{\linewidth}
  \begin{tikzpicture}
%  	\hspace*{20mm}
  	\pic [pic text={\LARGE \faLaptop}]  {cercle};
    \node[draw=none] at (0,-0.9) {Tecnologia};
    \pic [pic text={\LARGE \faLightbulbO}] at (26mm,0) {cercle};
    \node[draw=none] at (2.6,-0.9) {Inovação};
    \pic [pic text={\LARGE \faGroup}] at (51mm,0) {cercle};
    \node[draw=none] at (5.1,-0.9) {Voluntariado}; 
    \pic [pic text={\LARGE \faRocket}] at (76mm,0) {cercle};
    \node[draw=none] at (7.6,-0.9) {Ciência};   
    \pic [pic text={\LARGE \faBicycle}] at (101mm,0) {cercle};
    \node[draw=none] at (10.1,-0.9) {Ciclismo};
    \pic [pic text={\LARGE \faGithub}] at (126mm,0) {cercle};
    \node[draw=none] at (12.6,-0.9) {Open Source};
  \end{tikzpicture}
\end{minipage}

%  \vspace{2.1cm}
%  \begin{flushright}
%    \emph{Pedro Almeida}
%  \end{flushright}
%  \begin{flushright}
%    \textit{\today}
%  \end{flushright}

\end{document}